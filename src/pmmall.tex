\documentclass[11p]{article}
% Packages
\usepackage{amsmath}
\usepackage{graphicx}
\usepackage{fancyheadings}
%\usepackage{hyperref} % Fixar klickbara länkar
\usepackage[swedish]{babel}
\usepackage[
    backend=biber,
    style=authoryear-ibid,
    sorting=ynt
]{biblatex}
\usepackage[utf8]{inputenc}
\usepackage[T1]{fontenc}
% Källor
\addbibresource{mall.bib}
\graphicspath{ {} }

\def\name{Fabian Sigfridsson}
\def\email{fabian.sigfridsson@elev.ga.ntig.se}
\def\foottitle{PM Vindkraft}
\title{PM Vindkraft \\ \small Fysik 1}

\author{\name}
\date{\today}

\begin{document}

    \begin{titlepage}
        \begin{center}
            \vspace*{1cm}

            \Huge
            \textbf{ \title }

            \vspace{0.5cm}
            \LARGE
            Vindkraft

            \vspace{1.5cm}

            \textbf{ \author }

            \vfill

            Ett PM om energiförsörjning \\
            Fysik 1

            \vspace{0.8cm}

            \includegraphics[width=0.4\textwidth]{img.png}

            \Large
            Teknikprogrammet\\
            NTI Gymnasiet\\
            Umeå\\
            \today

        \end{center}
        \lfoot{\footnotesize{\name \\ \email}}
        \rfoot{\footnotesize{\today}}
        \lhead{\sc\footnotesize\foottitle}
        \rhead{\nouppercase{\sc\footnotesize\leftmark}}
    \end{titlepage}
% Om arbetet är långt har det en innehållsförteckning, annars kan den utelämnas
    \tableofcontents
    \newpage

    \section{Inledning}
    Vindkraften är en stor del av Sveriges energiförsörjning och sägs vara en helt förnybar energikälla, och vi ska undersöka den.

    \subsection{Frågeställningar}
    \begin{enumerate}
        \item Hur fungerar ett vindkraftverk?
        \item Vilka miljöpåverkan har ett vindkraftverk lokalt och globalt?
        \item Hur påverkar vindkraftverk samhället (Ekonomi/politik/konflikter/m.m.) lokalt och globalt?
    \end{enumerate}

    \section{Resultat}

    \subsection{Vindkraft, så fungerar det}
    Ett vindkraftverk fungerar genom att den omvandlar vindens rörelseenergi först till en mekanisk
    rotation som sedan går in i en växellåda som ökar rotationshastigheten och som där efter går in i en generator.
    Generatorn gör då om den mekaniska rörelseenergin till elektrisk energi \parencite[sid 267]{Fraenkel}.

    \subsection{Globala miljökonsekvenser av vindkraftverk}
    Vindkraftverk är både bra och dåliga för miljön. Dom är bra från ett energiproducerande perspektiv
    för att den genererar inte några växthusgaser efter att man har ställt upp och monterat den.

    \subsection{Lokal miljöpåverkan av ett vindkraftverk}
    Vindkraftverk är inte lika bra lokalt sett för att det krävs väldigt mycket resurser som släpper ut mycket växthusgaser.
    Det bullrar också och det behövs stora områden för vindkraftparker, Och efter att det har tagits ur bruk så lämnas det kvar väldigt mycket betong i marken.

    \subsection{Vindkraftverkens samhällspåverkan}
    Vindkraftverk är väldigt debaterade i dagens samhälle då det är en jättebra energikälla men också förstör miljön dom är i.

    \section{Slutsatser}
    Sammanfattat så kan man säga att vindkraftverk är en ganska neutral energikälla då den ger helt koldioxidfri energi,
    men samtidigt stör naturen runt om sig väldigt mycket.
    \printbibliography

\end{document}
